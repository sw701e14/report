\section{Clustering}\label{clustering}
In order to find the hotspot(s) defined in \Cref{hotspot}, we considered different clustering techniques.
This section will describe the concepts of clustering as well as the possible algorithms.
This section is based on \citet{pang2006introduction}.

\subsection{Cluster Analysis}
Cluster analysis is a technique used to group data objects based only on the information the data itself contains.
The requirement for membership in a certain cluster is often vague, as several acceptable clusterings can be made on the same dataset.
An example of this can be seen on \Cref{clusterings} where the same dataset has been clustered in three radically different ways, even though all three could be regarded as correct, depending on the purpose of the clustering.

The definition of a cluster therefore depends on the dataset and the purpose of the clustering 

\begin{figure}[H]
\includegraphics[trim= 3.1cm 20.61cm 5cm 2.5cm, clip=true]{graphics/different_clusters}
\centering
\label{clusterings}
\caption{Different clusterings on the same dataset. From \citet{pang2006introduction}.}
\end{figure}

\subsection{Types of clusterings}

A clustering is a collection of clusters.
This section will introduce the terminology used to describe clusterings. 

\paragraph{Hierarchical versus partitional}
A \textit{partitional clustering} is a division of data into pairwise disjoint subsets, where each object belongs to exactly one subset.
If clusters are allowed to have subclusters, the clustering is said to be \textit{hierarchical}.
A hierarchical clustering is represented as a set of nested clusters organized as a tree where each node is the union of its children.

\paragraph{Exclusive, overlapping and fuzzy clusterings}

A clustering is \textit{exclusive} if an object is assigned to exactly one cluster.
If an object can belong to more than one group the clustering is said to be \textit{overlapping}.
If a weight is used to describe the membership of sets the clustering is said to be \textit{fuzzy}.

\paragraph{Complete versus partial}

A \textit{complete} clustering has every object assigned to a cluster while a \textit{partial} clustering can have outliers that do not belong to any cluster.

\subsection{Types of clusters}
The idea of a cluster depends on the kind of data set it is applied on.
In this section the different notions of a cluster will be presented

\paragraph{Well separated}
A cluster is a set of objects where the objects that are similar are grouped in a cluster. 
Sometimes a threshold is used to define a minimum similarity. 
All objects in a cluster needs to be at least as similar to the other objects in a cluster for an object to be in the cluster.

\paragraph{Prototype based}
An object is placed in clusters based on prototypes that defines the clusters.
An object is placed in the cluster where the object is more similar to the prototype of the cluster than to the prototype of any other cluster.

These prototypes can be either the average value of a cluster or the most representative object of a cluster.

\paragraph{Graph based}
If the data can be represented as a graph with the objects as nodes, clusters can be defined as connected components in the graph.

\paragraph{Density based}
A cluster is defined by the density of the data objects.
A cluster is a dense region surrounded by a low density region.

\subsection{Our data and purpose}
The data we need to cluster is GPS data from the bikes.
A bike will send a GPS location in some interval, and it will be saved in a database.
We expect the points to be distributed in a wide area, but clustered around places where people place the bikes.
These clusters may have different size depending on the physical properties of the places the bikes are left.

The purpose of the clustering is to find the areas where the bicycles are being used the most in order to make predictions on when a bike will arrive in those areas.
A polygon will be used as a representation of a cluster, more about that in \Cref{convex_hull}.

Based on this description we need a hierarchical, exclusive, partial and density based clustering.

Hierarchically because we want to represent the clusters as a polygon.
This would ease the task of point location.
It needs to be exclusive because we want the clusters to be separated from each other so any point on the map is either in one cluster or not in any cluster.
We want to find out where the bikes spend most of their time and therefore we want a density based partial clustering.

\subsection{Techniques}
\bruno{I think we only should present DBSCAN and just mention the others with a cite.}
This section will explore the techniques that exist in cluster analysis and evaluate them based on the requirements stated in the previous section.

\subsubsection{K-means}
K-means is a technique that creates a partitioning from prototypes where the prototypes are the center of the clusters.

The basic algorithm takes a number K and generates K initial center points.
Each data object is then assigned to the nearest center point.
The center points are now updated to be the center of the created clusters.
These steps are repeated until no point changes cluster or the center points do not change.

Because we do not have a way of finding $ k $ before running the algorithm, this approach is not applicable to our problem.

\subsubsection{Hierarchical Clustering}
Hierarchical clustering arranges the points in a hierarchy depending on the distance between points.
Hierarchical clustering can be performed either by starting at the top of the tree or at the bottom.
\textit{Agglomerative hierarchical clustering} starts by considering all points as being in an individual cluster and generating the hierarchy by merging the closest clusters. 
\textit{Divisive hierarchical clustering} starts by considering all points as being in a single cluster and then continues by splitting the clusters at each step.

We have no need for a hierarchical division of our clusters, and this algorithm is therefor not appropriate for our use.

\subsubsection{DBSCAN}\label{clustering:DBSCAN}
DBSCAN in a density based clustering algorithm that locates regions of high density that are separated by regions of low density.

\paragraph{Center based point density}
There exists different methods of determining density in a set of data points.
One approach is to use a point as a center and calculate the number of points that are within a certain radius of the point.
Using this measure it can be classified whether a point is placed in a dense region (a core point),at the edge of a dense region (a border point) or is in a region with sparse density (a noise point).
The exact definition of the three types of points are as follows\cite{pang2006introduction}:
\\
\\
\noindent
\textbf{Core point} These points have at least $ MinPts $ points in a radius of $ eps $ from it.
Here $ MinPts $ is the number of points in the vicinity of a point regarded as dense by the user, and $ eps $ is the radius to look for these points.

\noindent
\textbf{Border points} A point that is not a core point, but is in the neighborhood of a core point. 
One point can be a border point to several core points.

\noindent
\textbf{Noise point} Any point that is neither a core point or a border point. 

\begin{figure}
\fbox{\includegraphics[trim= 3.1cm 20.61cm 5cm 2.5cm, clip=true]{graphics/DBSCAN_point_types}}
\caption{The three types of points in the clustering algorithm DBSCAN. Taken from \cite[page 528]{pang2006introduction}.}
\label{dbscan_point_types}
\end{figure}

A visual description can be seen on \Cref{dbscan_point_types}.
Given these definitions, the algorithm is described in \Cref{dbscan-algo}.

\begin{algorithm}
DBSCAN(D, eps, MinPts)\\
	C = 0\\
	\For {each unvisited point P in dataset D mark P as visited\\}
	{
		NeighborPts = regionQuery(P, eps)\\
		\eIf {sizeof(NeighborPts) < MinPts}{
			mark P as NOISE\\
		}
		{
			C = next cluster\\
			expandCluster(P, NeighborPts, C, eps, MinPts)
		}
	}
\caption{The DBSCAN clustering algorithm}\label{dbscan-algo}
\end{algorithm}

\begin{algorithm}
expandCluster(P, NeighborPts, C, eps, MinPts)\\
	add P to cluster C\\
	\For {each point Pn in NeighborPts\\}
	{
		\If{Pn is not visited\\}
		{
			mark Pn as visited\\
			NeighborPtsn = regionQuery(Pn, eps)\\
		}
		\If {sizeof(NeighborPtsn) >= MinPts\\}
		{
			NeighborPts = $ NeighborPts \cup NeighborPtsn $\\
			\If {Pn is not yet member of any cluster\\}
			{add Pn to cluster C\\}
		}
	}
\end{algorithm}%(Pn) is used instead of (p') in source (wiki).

\begin{algorithm}
regionQuery(P, eps)\\
	\Return {all points within the eps-neighborhood of P (including P)\\}
\end{algorithm}
\stefan{algortithm in appendix?}
\bruno{Den står fint her}

\subsection{Summary} Of the three presented algorithms only the DBSCAN algorithm fits our need for a exclusive, partial and density based clustering algorithm.
We therefore use this algorithm to detect clusters in our data for use as the hotspots in our model.
The hierarchical need is covered by saving the clusters in a tree data structure\footnote{This was not implemented, see \Cref{ref:ods}.}.