\section{Data input simulation}

In the running system data points will be collected from a GPS sender in the bikes.
Because we do not have a running system we will need to create a simulation of the behaviour the real system will have.

\subsection{The real system}
In the running system the data will be provided by a GPS sender and a receiver will interpret the data.
The sender will send at a set interval but it cannot be guaranteed that the interval will be consistent. 
There is also a chance that the sender will lose GPS signal, and thus it is not guaranteed that a point will be received at every interval.

The real system will consist of more than 100 bikes that send GPS points independent of each other.

\subsection{GPS receiver}
In order to accommodate for both the running system and our simulation we will construct an interface that handles incoming points.
This interface will be able to receive the points from any source so it will be easy to change the source from our simulated data to a real data source.

\subsection{Point generation}
Because we do not have access to a running system with 100 bikes sending GPS points, we have created a simulation of the real system.
This system creates GPS points at an interval and uses Google Directions to create realistic looking routes.
In order to generate points that are as realistic as possible we have used the following approach for generating GPS points.

\begin{itemize}
\item Generate a number of bike routes with the Google Directions API
\item Use the generated data points to generate the same route, but with a GPS point for each predefined interval.
\item Move each point a random distance at a random angle in order to simulate GPS precision.
\item 
\end{itemize}