\section{Data input simulation}\label{design:datasimulation}

In the running system data points will be collected from a GPS sender in the bikes.
Because we do not have a running system we will need to create a simulation of the behaviour the real system would have.

\subsection{The real system}
In the running system the data will be provided by a GPS sender and a receiver will interpret the data.
The sender will send at a set interval but it cannot be guaranteed that the interval will be consistent. 
It is also not to be guaranteed that the reported GPS locations will be exact.
This will be reported as the accuracy of the measurement by the GPS sender.

The real system will consist of more than 100 bikes that send GPS points independent of each other.

\subsection{GPS point receiver}\label{design:datareceiver}
In order to accommodate for both the running system and our simulation we will construct an interface that handles incoming points.
This interface will be able to receive the points from any source so it will be easy to change the source from our simulated data to a real data source.

The receiver handles every point that is coming from the bikes.
In order to decrease the number of data points as well as to make it easier to find bikes that are not moving, the receiver will analyse every incoming point and check if the bike has moved since the last update.
If the bike has not moved a boolean value is set on the previously recorded point in the database, and the incoming point will not be inserted in the database.
This check is done by comparing the distance between the two points and the accuracy of the data points, as reported by the GPS sender.
The algorithm used for this is simple and is displayed in \cref{withinacc}.
Here the distance function calculates the distance between the two locations contained in the GPS points and accuracy is a property of the GSP points.

\begin{algorithm}
\SetKwInOut{Input}{input}\SetKwInOut{Output}{output}
\Input{Two gps points gps1 and gps2}
\Output{True if the points can be regarded as the same due to accuracy, false if not}

dist = distance(gps1,gps2)\\
\Return gps1.accuracy + gps2.accuracy >= dist

\caption{WithinAccuracy}
\label{withinacc}
\end{algorithm}
\stefan{pseudo unnecessary?}

\subsection{Point generation}
Because we do not have access to a running system with 100 bikes sending GPS points, we have created a simulation of the real system.
This system creates GPS points at an interval and uses Google Directions to create realistic looking routes.
In order to generate points that are as realistic as possible we have used the following approach for generating GPS points.

\begin{itemize}
\item Generate a number of bike routes with the Google Directions API
\item Use the generated data points to generate the same route, but with a GPS point for each predefined interval.
\item Move each point a random distance at a random angle in order to simulate GPS precision.
\end{itemize}

The details of these steps will be explained in the following.

\subsubsection{Creating routes with Google directions}

The Google Directions API \cite{gdirections} have been used to create realistic looking bike routes.
The api has a variety of settings to restrict the results, including a setting to only get bike routes.
The result can be either XML or JSON and contains the route structured by a number of  ``legs'' which contains a number of ``steps''.
A leg is a part of a journey and a new leg will be created when changing means of transport or at waypoints.
For our purposes all routes will always contain only one leg.
Steps are the points where the route changes direction and are the points that would be used to print out a direction instruction for the route.
The result of a query to the Google directions API is a route that can be used by a human to find his way to his destination.

\subsubsection{Generating intervals}
The route generated by Google directions only has points where the route changes direction.
In order to make the generated routes look more like the data generated by a GPS sender there is generated points so that the GPS points are spaced out at a predetermined interval.
\stefan{the function is not particularly interesting, pseudocode?}

\subsubsection{Moving points to simulate precision}
The GPS senders are not 100\% reliable, and they will therefore not generate points that look like the ones generated by Google Directions.
To simulate this loss of precision every point generated will be moved a random distance at a random angle.
\stefan{pseudo?}