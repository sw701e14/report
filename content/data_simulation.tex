\section{Data Input Simulation}\label{design:datasimulation}
In the running system, locations will be retrieved from a GPS reciever attached on the bikes.
Because we do not have bikes with GPS receivers at our disposal, we need to create a realistic simulation of the behaviour of the system.

\subsection{Point generation}
Because we do not have access to a running system with X amount of bikes sending locations, we have created a simulation of the real system.
This system uses Google Directions to create pseudo-realistic routes and returns the locations at an interval.
In order to generate points that are as realistic as possible we have used the following approach for generating GPS points:

\begin{itemize}
\item Generate a number of bike routes with the Google Directions API
\item Use the generated locations to generate the same route, but with a \texttt{GPSData} for each predefined interval.
\item Perturb each \texttt{GPSData} in order to simulate GPS precision.
\end{itemize}
\mikkel{Det her er det første sted vi nævner GPSData - jeg synes ikke det fremgår af kontekst at det er en klasse i kode, og selvom det gør så ved man ikke hvordan den klasse ser ud.}

The details of these steps will be explained in the following.
\mikkel{Kunne vi ikke bare opstille de følgende subsubsections som en enumeration istedet?}

\subsubsection{Creating routes with Google directions}
The Google Directions API \cite{gdirections} has been used to create realistic looking bike routes.
The API has a variety of settings to restrict the results, including a setting to only get bike routes.
\mikkel{Jeg synes den første del om google API skal skiftes ud med hvad vi har gjort. Det er ikke så interessant hvad API'en kan, det er mere bare interessant at vi får output som cykel-router og i JSON. Det er ligemeget at vi kunne have fået andre ting.}
The result can be either XML or JSON and contains the route structured by a number of ``steps''.\footnote{Actually a route contains a number of  ``legs'' which contains a number ``steps''. 
A leg is a part of a journey and a new leg will be created when changing means of transport or at waypoints.
For our purposes all routes will always contain only one leg.}
Steps are the points where the route changes direction and are the points that would be used to print a directions instruction for the route.
The result of a query to the Google directions API is a route that can be used by a human to find his way to his destination.

\subsubsection{Generating intervals}
The route generated by Google directions only has points where the route changes direction.
In order to make the generated routes fit with the data generated by a GPS receiver, some points are added between two consecutive points, so that the generated GPS points are spaced out at a interval.
\mikkel{Det er ikke helt det der sker}
The generated GPS points are also delayed randomly.
Lost GPS points are not taken into account.

\subsubsection{Moving points to simulate precision}
GPS receivers are not 100\% reliable, and they will therefore not generate locations that look like the ones generated by Google Directions.
To simulate this loss of precision every point generated will be translated to a random distance at a random angle.
The amount, data points are moved, corresponds to the inaccuracy GPS receivers would have. This number is 15m according to \citet{garmingps}.
\mikkel{Jeg synes ikke vi giver udtryk for at alle punkterne har en accuracy som vi sætter til de 15m.}
