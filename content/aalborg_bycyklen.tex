Aalborg Bycyklen is a bike renting service located in Aalborg, Denmark.
It is a public service and available to all visiting and living in the city Aalborg.
The service is available from the 1st of may to sometime in the fall(not specified on their homepage).
\cite{aalborgbycyklenbagcyklen}

\paragraph{Components}
The service consists of the following components:
\begin{itemize}
\item Bikes, 
\item 21 bike stations with room for 170 bikes
\end{itemize}
The bikes have a lock that can be used at stations only.
The lock can be unlocked by putting in a coin(20 DKK) and when the bike is locked again the coin is released.

\paragraph{The renting flow}
The flow is pretty simple:
\begin{itemize}
\item The user is renting a bike at a station by unlocking it with 20 DKK coin
\item The user can cycle around.
\item The user is at his end of the tour and stops by a station to lock the bike and collect the 20 DKK back.
\end{itemize}

\subsection{Challenges}
This section will consider the challenges that Aalborg Bycyklen has.
The challenges described are from the groups own experience and perspective but still considered reasonable.

\paragraph{Bikes disappear}
The low deposit fee does that many users do not have the obligation to stop at a station near there destination.
Instead they cycle straight to their destination and leave the bike there.
The obvious problem is that when someone needs a bike they might not find one at a station.
Another problem is that Aalborg Bycyklen has to collect the cycles and return them to stations.

\paragraph{Flexibility}
The bikes can only be locked at stations.
This means that when a user cycles around and wants to make a stop where there is no bike station he just has to leave his bike.
Now he cannot lock the bike and risks that someone else takes the bike.
This means that the flexibility of Aalborg Bycyklen depends on the area to cover and the amount of bike stations.