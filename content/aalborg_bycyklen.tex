\section{\citybike}\label{aalborg_bycyklen}
\citybike is a communal bike service and therefore available to anyone living in or visiting the city of Aalborg.
The service is available from the 1st of May until the end of October \cite{aalborgbycyklenbagcyklen}.

\noindent The service consists of:
\begin{itemize}
\item 200 bikes (140 active) and 
\item 21 bike stations, with room for 170 bikes in total.
\end{itemize}
Each bike is provided with a lock, that enables locking a bike to a station (similar to shopping cart systems).
The bike is unlocked by depositing a coin (20 DKK) into the lock, after which it the bike is freely available for the user.
By returning the bike to a station, and re-applying the lock at a free stand, the deposited coin is returned.

\paragraph{Missing and broken bikes} are reported through a form on the \citybike website: \url{http://www.aalborgbycyklen.dk/}.

\paragraph{The general utilization of the service} can be summarized as follows:
\begin{enumerate}
\item A user unlocks a bike at a station, by inserting 20 DKK into a locked bike.
\item The user can freely use the bike around the city.
\item The user can return the bike at any station by locking the bike.
\end{enumerate}

\subsection{Problems in the current system} \label{aalborg_bycyklen:challenges}
This section will consider the challenges associated with \citybike.
The challenges described are devised from our own experience and perspective but still considered reasonable, as there is no target group according to the domain experts (Aalborg Kommune).
\mikkel{''but still considered reasonable'' - what does that refer to?}

%http://nordjyske.dk/nyheder/bycykler-ude-af-drift/e7a38ecd-3b1d-4290-8d2f-0f0e1a5d0fa6/4/1513
\paragraph{Bikes left outside stations}
The low deposit fee enables\alexander{enable vs allow} users to leave bikes, away from stations, without consequence other than the limited loss of 20 DKK.
It is possible to ride straight to another destination, outside a station, and leave the bike there.
The problem with this is that when someone else needs a bike, they might not be able to find one at a station, and without the possibility of knowing where they might be, outside of the stations.
Additionally, \citybike has to find the unused bikes outside stations, collect them, and return them to the stations.

\paragraph{Too few stations}
A ride is supposed to start and end at a station.
This, however, can be a problem when the user's actual start or destination is not close to a station.
Strictly speaking this shouldn't be possible, as it is not allowed to park a bike outside a station.
Practically speaking it is possible to do, and when done the problem as described in \textbf{Bikes left outside stations} emerges.

\paragraph{Making short stops}
The bikes can only be locked at stations.
This means that when a user, during a ride, wants to make a stop where there is no bike station, he has to leave his bike unlocked.
This entails a risk that a new user comes along and takes the bike (which now can be done without 20 DKK deposit, as it is not locked at a station), leaving the original user without a bike.
The original user then has to go to a station to acquire a new bike, hope to get lucky and find another bike nearby, or find another source of transportation from this point on.

\paragraph{No bikes at a station}
When users want to use a bike, they have to go to a station and see if there are any available bikes.
They have no way of knowing whether there is a bike at a station or not.
\mikael{For løst formuleret?}

\paragraph{No way of knowing when a bike will arrive}
If a user needs a bike at a specific time in a specific area, he can not be sure to find a bike at the location at the time.
\mikael{For løst formuleret?}

\paragraph{Broken bikes}
Aalborg Kommune does not know when a bike is in need of repair and rely solely on the users to report the whereabouts of broken bikes.
Potentially there could be bikes left outside the common-usage-area in need of repair. These bikes would be out of service for a long time.
