\subsection{Web Service API}
The web service API was implemented using ASP.NET Web API 2.3.\cite{aspnet_webapi}
ASP.NET Web API uses ASP.NET MVC\cite{aspnet_mvc}, with the use of \texttt{ApiController}, which is a specialization of the regular \texttt{Controller}.

\paragraph{Model} is simply used to describe the resource representations, as the real model is already found in \texttt{Data}'s model.
\mikael{Some kind of reference to architecture description.}
A resource is an object, containing named properties, along with sub-resources.
Sub-resources are represented as simple objects, containing only its relative URL.

\paragraph{View} can be seen as the serialized resource representation that the user obtains by performing requests against the API.
Per default, this is \texttt{text/xml}, however, by adding the \texttt{Accept} header, this can be set to \texttt{application/json}.

\paragraph{Controller} represents the handling of resources, which primarily consists of transforming the actual model, into the resources model, to be serialized and returned as response to requests.
The routing to controllers is handled by the attributes \texttt{[RoutePrefix]} and \texttt{[Route]}.
In order to handle the different HTTP methods, controller methods are also annotated with an appropriate \texttt{[Http\{Method\}]} attribute (e.g. \texttt{[HttpGet]}).

\paragraph{ROA} criteria, as described in \cref{webservice:roa}, will be followed.

\subsection{Resources}
% Command for displaying url, properties, and sub-resource of each resource
\newcommand{\resource}[5]{\begin{description}
\item[URL:]{\texttt{#1}}
\item[Properties:]{\texttt{#2}}
\item[Resources:]{\texttt{#3}}
\item[Methods:]{\texttt{#4}}
\item[Responses:]{\texttt{#5}}
\end{description}}

\resource{/}{version}{availablebikes, hotspots, bikes}{GET}{200 OK}
The root path serves as the entry point for the API.
It has a single property; version, to be used for asserting that the API has changes that implementations should react to.
More importantly, it contains a sub-resource for each main resource.

\noindent\hrulefill

\resource{/availablebikes}{count}{List of availablebikes}{GET}{200 OK}
This resource contains a list of all \texttt{availablebike} resources, along with a count of how many there are.

\noindent\hrulefill

\resource{/availablebikes/\{bikeId\}}{latitude, longitude}{}{GET}{200 OK, 404 Not Found}
This is the actual \texttt{availablebike} resource.
In case a client tries to request a bike that is not available, it will be met with a \texttt{404 Not Found} response.

\noindent\hrulefill

\resource{/hotspots}{hotspots}{}{GET}{200 OK}
\mikael{Has not been implemented.}

\noindent\hrulefill

\resource{/bikes}{count}{List of bikes}{GET}{200 OK, 401 Unauthorized}
\mikael{Has not been implemented.}

\noindent\hrulefill

\resource{/bikes/\{bikeId\}}{latitude, longitude, immobileSince}{}{GET}{200 OK, 401 Unauthorized, 404 Not Found}
\mikael{Has not been implemented.}
