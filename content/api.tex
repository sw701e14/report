\section{Web Service API}
The web service API was implemented using ASP.NET Web API 2.3.\cite{aspnet_webapi}
It follows the criteria set by ROA, as described in \cref{webservice:roa}.

Furthermore, the implementation uses ASP.NET MVC.
In this context, \textit{Model} is simply used to describe the resource representations, as the real model is already found in \texttt{Data}'s model.
\textit{View} is the serialized resource representation that the user can obtain by performing requests against the API.
\textit{Controller} represents the handling of resources, which primarily consists of transforming the actual model, into the resources to be returned as response to requests.

\subsection{Resources}

% Command for displaying url, properties, and sub-resource of each resource
\newcommand{\resource}[3]{\begin{description}
\item[URL:]{\texttt{#1}}
\item[Properties:]{\texttt{#2}}
\item[Resources:]{\texttt{#3}}
\end{description}}

\subsubsection{Root}
\resource{/}{version}{availablebikes, hotspots, bikes}
The root path \texttt{/} serves as the entry point for the API.
It has a single property; current version, and a sub-resource for each actual resource.

\subsubsection{Available Bikes}
\resource{/availablebikes}{count}{\{bikeId\}}
This resource contains a list of all available bike resources, along with a count of how many available bikes there are.

\paragraph{Available Bike}
\resource{/availablebikes/\{bikeId\}}{latitude, longitude}{}

\subsubsection{Hotspots}
\mikael{Not implemented yet}
\resource{/hotspots}{hotspots}{}

\subsubsection{All Bikes}
\mikael{Not implemented yet}
\resource{/bikes}{count}{{bikeId}}

\paragraph{Bike}
\resource{/bikes/{bikeId}}{latitude, longitude, immobileSince}{}