\subsection{Test Data}

For the project we needed some data to test our application.
Because the existing system does not use GPS we had to create our own data.
For this purpose we have chosen two methods of acquiring data.
These methods and their purpose will be described here.

\paragraph{FollowMee.com}
Followmee.com provides a service that tracks a phone like we expect the bikes will be tracked in the final system.
The gps data is usable to our purpose because it tracks a GPS device with realistic errors and accuracies.
An app reports to the site at a given interval and the data can then later be retrieved as a csv file.

The format of the csv file is as follows:

\noindent
\begin{tabular}{|l|}
\hline
\begin{tabular}{l l l l}
Date & Lat & Lng & mph  \\
2014-09-12 11:21:15 AM & 57.01163 & 9.99110&0 \\
\end{tabular}
\\
\hline
\hline
\begin{tabular}{l l l l}
\hline
km/h & Acc & Type & Date2\\
0&10&GPS&2014-09-12 11:21:15+02:00\\
\end{tabular}
\\
\hline
\end{tabular}

A member from the group had the app running for about a week to generate some somewhat realistic data for testing the accuracy of the gps.

\paragraph{Google Directions}
Google directions is an API provided by Google that can calculate the route between two points and output as a number of road fragments.
This provides a way of quickly testing specific test cases when debugging.
This data is usable to our purpose because it generates realistic routes, but the accuracies are not as with a real GPS device, and will eventually have to be simulated.

The API can output in either XML or JSON format.
An example of a very simple request of the route from a point to itself can be seen in \cref{gdir}.

\paragraph{Simulation}
As we had no running system to pull live data from, we need to use the two described methods for creating a simulation of a running system.
\stefan{describe our intentions of the simulation}