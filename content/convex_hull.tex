\section{Convex Hull}\label{convex_hull}
This section introduces convex hull and the theory is based on \citet[section 33.3]{aadbook}.

Given a set of points the smallest convex polygon\cite[33.1-5]{aadbook} is the convex hull.
The convex hull is represented by a set of points.
Graham-Scan\cite[page 1031]{aadbook}, an algorithm for finding the convex hull, has been implemented.
\bruno{Instead of citing, should we write more about it?}


\bruno{Should maybe be in a section for it self - for rounding of the chapter or like a section for: 'that is what we found out?'}
For each cluster the convex hull is found.
There is a certain loss of information by doing that, but it also saves some space.
But the project can easily be satisfied by this representation because the hotspots are only used to check whether there is a point in a hotspot or not.
The result, a set of points representing a polygon, is the representation of a hotspot.
