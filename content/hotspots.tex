Since we no longer have stations, bike users can now finish a ride anywhere.
This means we will potentially have a vast number of start/end locations.
This is a problem as locations very close to each other are considered to be different.

Instead we want to group these locations that are close to each other, and more frequent, into hotspots.
This way we limit the number of predictions needed, as we only model the most frequently travelled areas, as opposed to every single location.

In order to group the locations we explore clustering techniques, before choosing an appropriate one to implement.

Finally, in order to simplify hotspot representations, we apply convex hull, creating a polygon representation of a hotspot.

\section{Hotspot}\label{hotspot}
We define a hotspot to be an area with a closer density of points compared to the rest of the data.
The points considered have to be static to be a part of the hotspot meaning that we only consider points which has not moved since the last time interval.