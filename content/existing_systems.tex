INTRODUCTION

\subsection{Copenhagen: Bycyklen}
The overall purpose of Bycyklen\cite{cph_bycyklen} is to provide bikes for both single and repeated use.

A bike is then rented at an hourly rate of DKK25/hour.
In exchange for a fixed subscription of DKK70/month, the hourly rate is reduced to DKK6.
In both cases, an account must be created first.
This can be done in an internet browser or on the bike-mounted tablet.

Whenever a bike is to be used, all that needs to be done is log in with an existing account.
Then the bike is unlocked from the rack and the hourly charging begins.
The bike is then in use until it is placed at a rack again, logged out and locked.

It is also possible to reserve a bike, which in all cases cost DKK10.
This will hold a bike, by locking it, up to 45 minutes before it is needed.
Half an hour before the reservation time, a text is sent to inform whether a bike was successfully held or not.
A repeating reservation can be made.

As of now, there is a total of 20 stations and an estimated average of 15 bikes per station, giving a total of around 300 bikes.
The bike itself is electrically powered and equipped with GPS and a handlebar-mounted tablet.
The tablet is used for registration and signing in/out.
Additionally the tablet, with use of the GPS, can provide navigational guidance.

\subsection{New York City: Citi Bike NYC}
Citi Bike NYC\cite{nyc_citibike} is a system which makes available bikes for single use, but many times.

Unlike most other systems, you do not rent a single bike.
There are three methods of renting bikes:

\begin{itemize}
\item 24-hour pass: \$9.95
\item 7-day pass: \$25
\item Annual membership: \$95
\end{itemize}

With the 24-hour and 7-day passes, you can use a bike for 30 minutes at a time.
With annual membership this is extended to 45 minutes.
However, after each use, you can just acquire another bike.
If you do not return the bike after the time-period, overcharge fees occur.

In order to acquire a bike with a pass, a payment is made with a credit-card at a bike-station, this provides a single-use code.
This code is entered on the bike itself, which is equipped with a tablet solely for this purpose.
Along with the pass fee, a \$101 security hold is placed.
When a bike is returned and another bike is to be acquired, the credit-card is identified, it is not charged if a pass is already active, and a new code is provided.
For annual memberships a key is used to unlock bikes and no interaction with the bike-station is needed.
