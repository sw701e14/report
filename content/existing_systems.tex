Already existing systems are explored in order to draw inspiration from them.
They will also serve as guidelines to what should and shouldn't be repeated in a new system.

There are many public bike systems around the world, mainly in bigger cities.
The systems differ a lot in both how and where bikes are acquired and how the use of the bikes are charged.
The two overall methods lie in either a rent-system or an grab-if-available-system.
The rent-systems usually have very few stations, and bikes are rented for a longer period of time (days), similar to a car rental service.
Aalborg Bycyklen is an example of a grab-if-available-system.

Chosen for comparison are two systems; Bycyklen\footnote{Translation from Danish: 'The City Bike'}, located in Copenhagen and Citi Bike NYC, located in New York City.
These systems are similar to Aalborg Bycyklen, in that the bikes are publicly available, and at many different locations.
Rent-systems will not be considered, as these are too different from the more publicly available systems.

\subsection{Copenhagen: Bycyklen}
The overall purpose of Bycyklen\cite{cph_bycyklen} is to provide bikes for both single and repeated use.

A bike is rented at an hourly rate of 25 DKK/hour.
In exchange for a fixed subscription of 70 DKK/month, the hourly rate is reduced to 6 DKK.
In both cases, an account must be created first.
This can be done using an internet browser or the bike-mounted tablet.

Whenever a bike is to be used, all one needs to do is log-in with an existing account.
The bike is then unlocked from the rack and the hourly charges begin.
Once the bike is no longer in use and placed in a rack again, the user is logged out and the bike is locked.

%RETTET HERTIL - Drægert
It is possible to reserve a bike, which in all cases cost 10 DKK.
This will hold a bike, by locking it, up to 45 minutes before it is needed.
Half an hour before the reservation time, a text is sent to inform whether a bike was successfully held or not.
A repeating reservation can be made.

As of now, there is a total of 20 stations and an estimated average of 15 bikes per station, giving a total of around 300 bikes.
The bike itself is electrically powered and equipped with GPS and a handlebar-mounted tablet.
The tablet is used for registration and signing in/out.
Additionally the tablet, with use of the GPS, can provide navigational guidance.

\subsection{New York City: Citi Bike NYC}
Citi Bike NYC\cite{nyc_citibike} is a system which makes available bikes for single use, but many times.

Unlike most other systems, you do not rent a single bike.
There are three methods of renting bikes:

\begin{itemize}
\item 24-hour pass: \$9.95
\item 7-day pass: \$25
\item Annual membership: \$95
\end{itemize}

With the 24-hour and 7-day passes, you can use a bike for 30 minutes at a time.
With annual membership this is extended to 45 minutes.
However, after each use, you can just acquire another bike.
If you do not return the bike after the time-period, overcharge fees occur.

In order to acquire a bike with a pass, a payment is made with a credit-card at a bike-station, which provides a single-use code.
This code is entered on the bike itself, which is equipped with a tablet solely for this purpose, which will unlock the bike from the bike-rack.
Along with the pass fee, a \$101 security hold is placed.
When a bike is returned, and another bike is to be acquired, the credit-card must be used again.
The credit-card will not be charged again if a pass is already active, instead a new code is provided.
For annual memberships a key is used to unlock bikes and no interaction with the bike-station is needed.

The exact number of bikes and stations is unknown.
The official website states that there are ''100s of stations'' and ''1000s of bikes''.
The bikes are very simple, with no motor or GPS, and the tablet in front is only used for unlocking bikes with either pass-code or membership-key.