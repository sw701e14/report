\section{Making predictions}
As a Markov chain $\mathcal{M}$ can be represented with a matrix $\mathbf{M}$ (constructed from \Cref{algo:markov}), we can use $\mathbf{M}$ to give an estimate of the behavior of the average bike in future time-steps.
For this we require an initial probability distribution for the average bike.
We can then apply $\mathbf{M}$ to this distribution to determine the probability of the average bike being in various states after a number to time-steps.
In this section, we will describe how this initial state is established and how we use it to handle predictions.

It should be noted, that the use of the term \textit{''prediction''} in this section refers to determining possible future states of a system, estimated by the Markov chain model.

\subsection{Initial state}
