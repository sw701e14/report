\section{Data Flow}
This section is about the data flow. More precisely how the data that is received from the GPS receiver is send to the database and what it is used for.
A diagram of the flow can be seen below on \cref{fig:dataFlowDiagram}.
The diagram is explained in the following.

\begin{figure}[H]
\includegraphics[width=\textwidth]{dataflowdiagram.pdf}
\caption{The data flow diagram of our solution}
\label{fig:dataFlowDiagram}
\end{figure}
\pagebreak

\paragraph{Database}
The database is explained \cref{} \bruno{ref når det er skrevet.}

\paragraph{From raw location data to \texttt{GPS data}}
Each bike is equipped with a GPS receiver which uses this to calculate its location.
The data is then send containing the location and a number indicating the accuracy of the location.
The data received is read and transformed to an object.
The object is inserted into the database.

\paragraph{From \texttt{GPS data} to Hotpots}
The \texttt{GPS data} are available from the database and is used by the clustering algorithm DBSCAN(\cref{clustering:DBSCAN}) to find all clusters.
By using convex hull (\cref{} \bruno{ref til convex hull - hvis vi skriver om det ellers bare en ekstern ref}) on each cluster the amount of \texttt{GPS data} is greatly minimized and is saved as a \texttt{Hotspot}.
The \texttt{Hotspot}'s are then saved to the database.

\paragraph{From Hotspots to Markov Chain}
This is building the Markov Chain(\cref{markov}) from the routes and \texttt{Hotspot}'s stored in the database.
The Markov Chain is then saved in the database.