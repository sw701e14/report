\section{Integration test}

In order to test that our components interact together according to the specifications, we have chosen to perform integration testing of the system.

Integration testing has the purpose of testing that two units work together. 
Traditionally, integration testing is done by integrating two modules at a time, either from the top of the dependency tree or from the bottom.
There exists a variant, in which the whole system is tested together, instead of pair-wise, called big bang integration testing.
We have chosen to perform our test with this method, due to time limitations \cite{inttest}.

\subsection{Test setup}
The purpose of the tests we have designed is to check if the system can produce sensible predictions for the bikes in the system.

In order to generate test data, we have created a method for simulating real world data. 
The simulator gets a list of destinations (in the form of addresses) as input, making it possible to adjust the probability of the system, by deliberately choosing certain destinations.
This makes it possible to create a data set, where it is possible to predict the expected result.

\paragraph{What to test}
We want to test if the system works as specified in the problem statement.
The points we want to focus on testing are \ref{pr_few}: Too few stations and \ref{pr_arrive}: No way of knowing when a bike will arrive. 
The full problem statement can be found in \cref{problem_statement}.
We want to test the following parts of our system in order to test the two problem statement items:

\begin{itemize}
\item Finding hotspots 
\item Markov Chains
\end{itemize}

We test the hotspot creation in order to check if hotspots are generated where there is bike activity, and this to check if the created hotspots will work as a stand in for the old stations.
This corresponds to item \ref{pr_few} in the problem statement.

We test the markov chains in order to figure out if our method of generating markov chains give proper values for using when predicting the behavior of the bikes.
This corresponds to item \ref{pr_arrive} in the problem statement.

\subsection{Finding hotspots}
The system relies on the clusters to be indicative of how the bikes are used. 
All predictions uses the hotspots as a basis and misplaced hotspots can therefore skew the results, degrading the precision of the predictions.
We therefore want to test if the hotspots generated are as they are expected.

\paragraph{Test data and cases}
For this test, a data set has been created.
It consists of five addresses and their corresponding GPS coordinates.
We will then compute the hotspots based on this data and compare the center of these hotspots with the addresses they are generated from.
Because the addresses are the only place where the bikes stand still, there should only be clusters by the addresses.

Three of the points are well separated, in order to test the general cluster creation.
The last two points are placed one street from each other, in order to test how the clustering handles points that are near each other.

\subsection{Predictions}
The users rely on the predictions of bike usage, as stated in \Cref{prob_statement:solution}, item 5.
It is therefore important to test how well the predictions work.

\subsubsection{Test data} To test this we create a number of controlled data sets.
The data sets are controlled by assigning probabilities of choosing the destinations. 
We then know where to expect the probabilities to be high.

\paragraph{Even distribution}
The obvious test case is when the generated data has equal probability of choosing every destination. 
In this case we would expect the probabilities of a bike arriving at every destination to be the same.

The data was generated to simulate a 15 hour period with four possible addresses and 100 bikes.
To control the data, every time a bike arrived at a hotspot it would wait for the next destination for 12 minutes.
The ensures that the is registered a point that gets the "hasNotMoved" flag, because it stays in the hotspot for two cycles.
The results for the test can be seen in \Cref{test_even}.
The table shows the markov chain created on the generated data, after four clusters was created from the points marked "hasNotMoved".
H indicates a hotspot while D indicates a departure state.
The number in $ (h_1,d_2) $ indicates the probability of transitioning from hotspot 1 to departure state 2 in percent. 

The addresses are placed so it should not be possible to go from one hotspot to another in a single update interval.
This means that it should only be possible to get to a hotspot from a departure state, and not directly.
This means that every hotspot row in the result should contain only two numbers, the transition to itself and the transition to its departure state.
In the result it is seen that some inaccuracies have occurred since there are some entries with values that, although very low, are not zero as expected.

The important numbers are those that tell how a bike act when in a departure state.
These are the numbers that indicate how the routes are expected to be chosen.
These numbers are pairwise marked with a color in the table.
Each of these sets of numbers indicate the distribution between destinations.
As the distribution of the generated data was set to be equal, we would expect these numbers to be equal as well.

\begin{figure}
	\centering
\begin{tabular}{|c | c c c c c c c c|}
\hline
\% &      $ h_1 $ & $ d_1 $ & $ h_2 $ & $ d_2 $ & $ h_3 $ & $ d_3 $ & $ h_4 $ & $ d_4 $\\
 \hline
$ h_1 $ & 59,6 &  40,1 & 0,3 &   &   &   &   &  \\
$ d_1 $ & 0,1 &  74,6 &  { \color{red} 8,6} &   &   {\color{red}8,3} &   &  {\color{red} 8,5} &  \\
$ h_2 $ & 0,2 &   &  59,6 &  40,1 &   &   &   &  \\
$ d_2 $ & {\color{blue}9,8} &   &   0,1 &  68,4 &  {\color{blue}11,4} &   &  {\color{blue}10,4} &  \\
$ h_3 $ & &   &   0,1 &   &  59,2 &  40,7 &   &  \\
$ d_3 $ & {\color{orange}12,4} &   &  {\color{orange}13,4} &   &   0,2 &  60,4 &  {\color{orange}13,7} &  \\
$ h_4 $ & &   &   &   &   &   &  60,0 &  40,0\\
$ d_4 $ & {\color{purple}7,7} &   &   {\color{purple}6,8} &   &   {\color{purple}6,9} &   &   0,0 &  78,5\\
\hline
\end{tabular}
\caption{Results of the test with an even distribution}\label{test_even}
\end{figure}

The numbers are normalized so they add up to 100 \% and can be seen on \cref{even_results} where the expected results are in the left table and the actual results are in the right table.

\begin{figure}
	
	\begin{tabular} {c c }
		
		\begin{tabular}{ | c | c c c c |}
			\hline
			& $ h_1 $ & $ h_2 $ & $ h_3 $ & $ h_4 $\\
			\hline
			$ d_1 $ & $ d $ & 33.33 & 33.33 & 33.33\\
			$ d_2 $ & 33.33 & $ d $ & 33.33 & 33.33\\
			$ d_3 $ & 33.33 & 33.33 & $ d $ & 33.33\\
			$ d_4 $ & 33.33 & 33.33 & 33.33 & $ d $\\
			\hline
		\end{tabular}
		
		&
		
		\begin{tabular}{ | c | c c c c |}
			\hline
			& $ h_1 $ & $ h_2 $ & $ h_3 $ & $ h_4 $\\
			\hline
			$ d_1 $ & $ d $ & 33.85 & 32.68 & 33.46\\
			$ d_2 $ & 31.01 & $ d $ & 36.08 & 32.9\\
			$ d_3 $ & 31.39 & 33.92 & $ d $ & 34.68 \\
			$ d_4 $ & 35.98 & 31.78 & 32.24 & $ d $\\
			\hline
		\end{tabular}
	\end{tabular}
	\caption{The expected proportions to the left and the resulting proportions to the right}\label{even_results}
\end{figure}

As it can be seen from the results, the numbers look very reasonable considered that they were randomly generated. 
The biggest deviation is the transition from $ d_4 $ to $ h_1 $ which is 7.95\% deviation. \stefan{kan man konkludere mere på det?}

\paragraph{Uneven distribution}
Another test case is to weight one destination higher than the others.
In this case we would expect the probability of a bike arriving at this particular destination to be higher than all the other probabilities.

The data for this test was created in the exact same way as the data for the even distribution, a 15 hour period with four possible addresses and 100 bikes.
The only difference was that the distribution was changed so one address had 50 \% chance, another had 20 \% and the two remaining had 15 \% chance of being chosen.

The resulting markov chain after this simulation can be seen on \Cref{test_uneven}.

\begin{figure}
	\centering
	\begin{tabular}{|c | c c c c c c c c|}
		\hline
		\% &      $ h_1 $ & $ d_1 $ & $ h_2 $ & $ d_2 $ & $ h_3 $ & $ d_3 $ & $ h_4 $ & $ d_4 $\\
		\hline
		$ h_1 $ & 60,4 &  39,6 &   0,1 &   &   &   &   &  \\
		$ d_1 $ & 0,1 &  59,4 &  {\color{red}24,3} &   &   {\color{red}6,9} &   &   {\color{red}9,4} &  \\
		$ h_2 $ & 0,2 &   &  59,3 &  40,4 &   &   &   &  \\
		$ d_2 $ & {\color{blue}7,4} &   &   0,2 &  74,8 &   {\color{blue}7,6} &   &  {\color{blue}10,1} &  \\
		$ h_3 $ & &   &   &   &  59,9 &  40,1 &   &  \\
		$ d_3 $ & {\color{orange}4,0} &   &  {\color{orange}11,4} &   &   &  80,1 &   {\color{orange}4,5} &  \\
		$ h_4 $ & 0,1 &   &   &   &   &   &  59,0 &  40,9\\
		$ d_4 $ & {\color{purple}7,4} &   &  {\color{purple}22,8} &   &   {\color{purple}5,7} &   &   0,1 &  64,1\\
		\hline
	\end{tabular}
	\caption{Results of the test with an uneven distribution}\label{test_uneven}
\end{figure}

It is not possible to translate the input addresses to indexes in the markov chain because the markov chain is based on the clusters found by the DBSCAN algorithm.
Based on the colored numbers it seems that $ h_1 $ had 15\% chance, $ h_2 $ had 50\% chance, $ h_3 $ had 15\% chance and $ h_4 $ had 20\% chance of being chosen.
\todo{bedre at plotte hotspots og vurdere derudfra?}

Based on these hotspot assignments we would expect the following proportions for the departure rows, where the $ d $ in each row is any number determining how long a bike is in a departure state before it leaves.
The numbers are normalized so they add up to 100 \% and can be seen on \cref{uneven_results} where the expected results are in the left table and the actual results are in the right table.

\begin{figure}

\begin{tabular} {c c }

\begin{tabular}{ | c | c c c c |}
	\hline
	 & $ h_1 $ & $ h_2 $ & $ h_3 $ & $ h_4 $\\
	\hline
	$ d_1 $ & $ d $ & 58.82 & 17.65 & 23.53\\
	$ d_2 $ & 30 & $ d $ & 30 & 40\\
	$ d_3 $ & 17.65 & 58.82 & $ d $ & 23.53\\
	$ d_4 $ & 18.75 & 62.5 & 18.75 & $ d $\\
	\hline
\end{tabular}

&

\begin{tabular}{ | c | c c c c |}
	\hline
	& $ h_1 $ & $ h_2 $ & $ h_3 $ & $ h_4 $\\
	\hline
	$ d_1 $ & $ d $ & 59.85 & 17.00 & 23.15\\
	$ d_2 $ & 29.48 & $ d $ & 30.28 & 40.24\\
	$ d_3 $ & 20.10 & 57.29 & $ d $ & 22.61 \\
	$ d_4 $ & 20.61 & 63.51 & 15.88 & $ d $\\
	\hline
\end{tabular}
\end{tabular}
\caption{The expected proportions to the left and the resulting proportions to the right}\label{uneven_results}
\end{figure}

These results are a little more difficult to judge based on the tables, and a table with the deviations for each of the results have therefore been constructed in \cref{deviations}.

\begin{figure}
	\centering
	\begin{tabular}{ | c | c c c c |}
	\hline
	& $ h_1 $ & $ h_2 $ & $ h_3 $ & $ h_4 $\\
	\hline
	$ d_1 $ & $ d $ & 1.75 & 3.68 & 1.61\\
	$ d_2 $ & 1.73 & $ d $ & 0.93 & 0.6\\
	$ d_3 $ & 13.88 & 2.60 & $ d $ & 3.01 \\
	$ d_4 $ & 9.92 & 1.62 & 15.30 & $ d $\\
	\hline
\end{tabular}
\caption{Deviations for the results of the unevenly distributed test} \label{deviations}
\end{figure}

The biggest deviations are the transitions from $ d_3 $ to $ h_1 $ and from $ d_4 $ to $ h_3 $ on 13.88\% and 15.30\% respectively.

These deviations are higher than the results for the evenly distributed test, but it is still reasonable considered the randomly generated data.

\subsubsection{Conclusion}
This section has tested the validity of the markov chains that are being used in our system.
The results showed that there is up to 15\% deviation from the expected numbers, but considering that the data was randomly generated we consider this in an acceptable range.
