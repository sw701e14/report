\section{Integration test}

In order to test that our components interact together as we would expect, we have chosen to do an integration test.

Integration testing has the purpose of testing that two units work together. 
Traditionally, integration testing is done by integrating two modules at a time, either from the top of the dependency tree or from the bottom.
There exists a variant, in which the whole system is tested together, instead of pair-wise, called big bang integration testing.
We have chosen to perform our test with this method, due to time limitations. \cite{inttest}

\subsection{Test setup}
The purpose of the test is to figure out if the system can produce sane predictions for the bikes in the system.

\stefan{coupling with problem statement}

In order to generate test data, we have created a method of simulating real world data. 
The simulator gets a list of destinations (in the form of addresses) as input, making it possible to adjust the probability of the system, by deliberately choosing certain destinations.
This makes it possible to create a data set, where it is possible to predict the expected result.

\paragraph{What to test}
We want to test the following parts of our system:

\begin{itemize}
\item Finding hotspots
\item Predictions
\end{itemize}

\subsection{Finding hotspots}
The system relies on the clusters to be indicative of how the bikes are used. 
All predictions uses the hotspots as a basis and misplaced hotspots can therefore skew the results, degrading the precision of the predictions.
We therefore want to test if the hotspots generated are as they are expected.

\paragraph{Test data and cases}
For this test, a data set has been created.
It consists of five addresses and their corresponding GPS coordinates.
We will then compute the hotspots based on this data and compare the center of these hotspots with the addresses they are generated from.
This will give an indication of how well our hotspots are created.

Three of the points are well separated, in order to test the general cluster creation.
The last two points are placed one street from each other, in order to test how the clustering handles close points.
\stefan{ikke sikker på om den del giver mening}

\subsection{Predictions}
The users rely on the predictions of bikes to be somewhat telling of how the bikes move around in the city.
It is therefore important to test how well the predictions work.

\paragraph{Test data} To test this we create a number of controlled data sets.
The data sets are controlled by assigning probabilities of choosing the destinations. 
We then know where to expect the probabilities to be high.

\paragraph{Test cases}
The obvious test case is when the generated data has equal probability of choosing every destination. 
In this case we would expect the probabilities of a bike arriving at every destination to be the same.

Another test case is to weigh one destination higher than the others.
In this case we would expect the probability of a bike arriving at this particular destination to be higher than all the other probabilities.