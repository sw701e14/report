\section{Integration test}

In order to test that our modules interact together as we expect we have chosen to do an integration test.

Integration testing has the purpose of testing that two units work together. 
Traditionally integration testing is done by integrating two modules at a time either from the top of the dependency tree or from the bottom.
There exists a variant which tests the whole system together from the beginning, called big bang integration testing.
We have chosen to perform our test with this method because of time limitations. \cite{inttest}

\subsection{Our test setup}
The purpose of the test is to figure out if the system can produce sane predictions for the bikes in the system.

We have made a system that can simulate the real world data. 
The simulator gets a list of destinations as input and it is then possible to adjust the probability of the system choosing a certain destination. 
This makes it possible to create a data set where it is possible to predict the expected result.

The obvious test case is when the generated data has equal probability of choosing every destination. 
In this case we would expect the probabilities of a bike arriving at every destination to be the same.

Another test case is to weigh one destination higher than the others.
In this case we would expect the probability of a bike arriving at this particular destination to be higher than all the other probabilities.