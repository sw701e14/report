The following section will describe the possibilities when designing a web service, arguing for our choice of web service.
After the chosen architecture has been presented, the actual implementation will be presented.

\section{Choosing an architecture}
We want to use a webservice to make the content of our model available to the users of the system.
When a user wants to know if a bike is available in their vicinity they will need a way to retrieve information about the bikes in the model.

REST\footnote{Representational State Transfter} is an architecture that exposes resources and as a bike can be represented as a resource REST will be our choice of web service architecture.

The remainder of this section will describe REST web services, specifically Resource-Oriented Architecture, based on \citet{restful_web_services}, \citet{fielding_dissertation}, and HTTP/1.1 Specification\cite{http_specification}.


\subsection{Representational State Transfer(REST)}
REST is an architectural style, which provides design criteria for network-based software architectures.
Web services adhering to these design criteria are referred to as RESTful web services.
These criteria were defined in \citet[Chapter 5]{fielding_dissertation} and are listed here:
\begin{enumerate}
\item Client-Server
\item Stateless
\item Cache
\item Uniform Interface
\item Layered System
\item Code-On-Demand (optional)
\end{enumerate}

Since REST is only an architectural style, making it possible to implement various correct architectures, we will be following one specific architecture implementation: Resource-Oriented Architecture (ROA), as described in \citet{restful_web_services}.

\subsection{Resource-Oriented Architecture(ROA)}\label{webservice:roa}
ROA is an architecture, upholding the REST design criteria.
This is done by organizing and accessing resources in a standardized manner, and by following the HTTP/1.1 specification.

In order to ensure that a ROA implementation will comply with the REST design criteria, it will have to adhere to the four concepts\cite[Chapter 4]{restful_web_services}:
\begin{itemize}
\item Resources
\item URIs
\item Representing resources
\item Linking resources
\end{itemize}
along with the four properties:
\begin{itemize}
\item Addressability
\item Statelessness
\item Connectedness
\item Uniform interface
\end{itemize}.
\begin{description}
\item[Resources] A resource is a subset of the resource state.
A single resource is an entity important enough to be a thing in itself.
Each resource should have at least one \textit{URI} (Unique Resource Identifier).
URIs ensure that each resource can be uniquely identified, directly handling the \textit{addressability} property.

\subparagraph{Representations} A representation of a specific resource is a collection of any useful information about that resource's state, defined by properties and sub-resources.
Representations of resources depends on the desired use.


\item[Connectedness and linking resources] As each resource is uniquely identified, any resource containing sub-resources or needing to refer to other resources, can do so by linking to its URI.
Thus, connectedness and linking resources are highly dependent on URIs.

\item[Statelessness] By providing, if any, state-relevant parameters on each request, statelessness is achieved.
This makes the server independent of clients and previous requests, as it is up to each individual client to handle its own application state (not to be confused with the server's resource state) and providing it when needed.

\item[Uniform interface] is where the HTTP/1.1 specifications come in.
HTTP provides a set of methods for retrieving and modifying resources.
As the service provided by \projectname{} will only allow users to retrieve data we will on be employing the GET method.
All resources can then be retrieved using a GET request (see \Cref{webservice:resources} for a specification).
Any feedback needed to be given to clients from the server is handled through HTTP status codes (e.g. '200 OK' and '404 Not Found').
\end{description}