\section{Web Service}
In order to increase availability, but without having to implement the system on all platforms ourselves, we can define and create a web service.
This will make it possible for others to use the web service, in order to create actual platform implementations.

When designing a web services, there are two overall architecture styles; REST and RPC.
Before choosing a solution, the two possiblities will be shortly explored.
The following sections will be based on \citet{restful_web_services}.

\subsection{REST}
Representational State Transfer (REST) is an architectural style, which provides guidelines for what are called RESTful web services.
Most important are that they use the HTTP protocol\cite{http_specification} and take advantage of the HTTP-provided methods (GET, POST, PUT, DELETE, etc).

\paragraph{Resource-Oriented Architecture (ROA)} is meant as a standardisable way of accessing resources through an easy-to-understand interface.
Resources are made available through URLs, where each URL points to a specific resource (or collection of same-type resources).
However, in order to generalize the access of similar resources, scoping information can be put in the URL.
What you want to do with the resource is determined by the HTTP method.

\subsection{RPC}
Remote Procedure Call (RPC) is a procedure-oriented architectural style.
RPC-style architectures usually make use of only one HTTP method; POST.
The actual method information is supplied in the entity body of the request.

\paragraph{SOAP}
SOAP is a well-defined RPC-style web service architecture.
It implements XML-RPC\cite{xmlrpc_specification} and traditionally uses HTTP for messaging, but can also use other protocols such as Secure Mail Transfer Protocol (SMTP).
It has a heavy use of XML schemas for defining the headers and bodies of its messages.
For instance, it uses Web Service Definition Language (WSDL), which states how the service is to be used; which procedures are available and which parameters must be supplied.

