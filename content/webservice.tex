\section{Web Service}
As we have very little knowledge of our target group, we would want our system as available as possible.
One way to do this, is to implement it on a multitude of platforms, e.g. PC, Android devices, iOS devices.
However, this would require massive work, as all these platforms are very different and would therefore require very different implementations.

Another way to make the system available, but without implementing it on all these platforms, would be to create a web service, for other developers (or hobbyists) to use, making their own platform implementations.

This section will shortly describe the possibilities when designing a web service, before arguing why a RESTful web service is the correct choice for our system.

\subsection{Types of web services}
When designing a web services, there are two overall architectural styles; REST and RPC.
The following sections will be based on \citet{restful_web_services}.

\subsubsection{REST}
Representational State Transfer (REST) is an architectural style, which provides design criteria for what are called RESTful web services.
These criteria were defined in \citet{fielding_dissertation}.
However, these criteria are in no way enforced, making it possible to design web services that are intended to be RESTful, but actually might not be.

\paragraph{Resource-Oriented Architecture (ROA)} is an architecture, upholding the aforementioned design criteria.
This is done by organizing and accessing resources in a predefined manner.
Resources are made available through URLs, where each URL points to a specific resource.
If a resource has resources of its own, this generates a sub-URL for each sub-resource.
In order to generalize the access of same-type resources, but wanting to filter the result, scoping information can be put in the URL as parameters.
What you want to do with the resource(s) is determined by the HTTP method (e.g. GET, PUT).

\subsubsection{RPC}
Remote Procedure Call (RPC) is a procedure-oriented architectural style.
RPC-style architectures usually make use of only one HTTP method; POST.
The actual method information is supplied in the entity body of the request.

\paragraph{SOAP} is a well-defined RPC-style web service architecture.
It implements XML-RPC\cite{xmlrpc_specification} and traditionally uses HTTP for messaging, but can also use other protocols, such as Secure Mail Transfer Protocol (SMTP).
It has a heavy use of XML schemas for defining the headers and bodies of its messages.
For instance, it uses Web Service Definition Language (WSDL), which states how the service is to be used; which procedures are available and which parameters must be supplied to those procedures.
