\section{Interview with Aalborg Kommune}
The \citybike project is maintained by Aalborg Kommune.
In order to get information about \citybike, an interview was conducted with three people involved with the project.

The interview was carried out October 10th, 2014, as a semi-structured interview.
The interviewees were:
\begin{itemize}
\item Brian Høj; Responsible for the \citybike project.
\item Jesper; Member of the \citybike project.\mikkel{What does it mean to be a member of the project?}
\item Anne Mette; Member of the original team which implemented the CIVITAS-ARCHIMEDES project\cite{aalborgbycyklenbagcyklen}.
\end{itemize}

\noindent There were two overall objectives of the interview:
\begin{enumerate}
\item To gather information about the current system.
\item To determine if the project members themselves already had improvements in mind.
\end{enumerate}

The information inquired about was:
\begin{itemize}
\item How is the system used?
\item Who uses the system?
\item What are the shortcomings of the system?
\end{itemize}

\mikkel{Visual again... I think that this section looks very sparse with all the enumerations. Could we do without some of them?}

\subsection{Results} \label{interview:goals}
The following section is based on the Danish summary of the interview, found in \Cref{interviewReferat}.

\paragraph{Overall objective}
The intended use of the bikes are single, short trips.
This means that the bikes are not meant to be used for longer periods of time, or to completely replace a personal bike.

\paragraph{Users and usage}
There is no target group for the usage of the bikes.
The bikes are currently not tracked so they were not able to say anything about the actual usage.
The only information they have is what they themselves observe when they walk around town and what citizens report.
\mikkel{What do they observe and what is reported?}

\paragraph{Future plans}
The interviewees have thought about installing GPS chips in the bikes before, to receive data about their movement.
They would really like to possess an overview of the bike usages and usage patterns.
Specifically, they would like some information about:
\begin{itemize}
\item What routes are travelled?
\item How long, both in distance and time, are the trips?
\item When are bikes used?
\item Who uses the bikes?
\end{itemize}

\paragraph{Rent/Booking}
It has never been the intention of the \citybike project to be able to rent or book a bike, because it would hamper the spontaneous use of the bikes.
If all the problems with booking could be solved in a reasonable manner, they would not rule out the future of such a system.

\paragraph{Missing bikes}
According to Traffic \& Roads, Aalborg Kommune\cite{cykelplanlaegning} there was a 11\% loss of city bikes from 2009 to 2011 (237 to 210 bikes).
According to the interviewees there now, fall 2014, remain 200 of the original 237 bikes from summer 2009.
The interviewees see this as such a small reduction in bikes, that missing and stolen bikes are not considered a problem.

\subsection{Conclusion}
The interview yielded the following points which will be used in deciding which problems to focus on.

\paragraph{No renting/booking system}
This was deemed a bad idea, as it opposes the intention of keeping the system simple and available.

\paragraph{No specific target user group}
Aalborg City Bike states that there is no target group.

\paragraph{Statistics about usage}
\citybike would like some statistics about how the bikes are used; e.g. which routes are travelled.
They also liked the idea of adding GPS receivers to the bikes in order to achieve this.

\paragraph{Short period usage}
The intended use of the system is short trips.

\subsubsection{Further contact}
After the interview we tried to contact Aalborg Kommune again, but despite several attempts, it was not possible to come into contact with them.
