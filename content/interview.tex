\section{Aalborg Kommune}
Aalborg Bycyklen is maintained by Aalborg Kommune.
In order to get information about the current Aalborg Bycyklen system, as well as ideas for improvements, an interview was conducted with three people involved with the project.

The interview was carried out November 10th, 2014, as a semi-structured interview.
The interviewees were:
\begin{itemize}
\item Brian Høj; Responsible for the Aalborg Bycykel project.
\item Jesper; Employee at Aalborg Bycykel.
\item Anne Mette; Member of the original team which implemented the CIVITAS-ARCHIMEDES project.
\end{itemize}

The following section is based on the Danish resume of the interview, found in \cref{interviewReferat}.

\subsection{Interview}

\paragraph{Goals and needs} \label{interview:goals}
The intentioned use of the bikes are single, short trips.
This means that the bikes are not meant to be used for longer periods of time, or to completely replace a personal bike.

\paragraph{Users and usage}
There is no target group for the usage of the bikes.
The bikes are currently not tracked so they were not able to say anything about the actual usage.
The only information they have is what they themselves observe when they walk around town and what citizens report.

\paragraph{Future plans}
The interviewees have before thought about installing GPSs in the bikes to receive data about their traffic.
They would really like to have an overview of the bike usages and usage patterns.
Additional they would like some information about:
\begin{itemize}
\item What routes are traveled?
\item How long a distance has been travelled in this trip?
\item When was the bike used?
\item Who uses the bikes?
\end{itemize}

\paragraph{Rent/Booking}
It has never been the intention of the project to be able to rent or book a bike, because it would hamper the spontaneous use of the bikes.
If all the problems with booking could be solved in a reasonable manner, they would not rule out the future of such a system.

\paragraph{Missing bikes}
According to Traffic \& Roads, Aalborg Kommune\cite{cykelplanlaegning} there was a 11\% loss of bikes from 2009 to 2011 (237 to 210 bikes).
According to the interviewees there now, fall 2014, remain 200 of the original 237 bikes from summer 2009.
The interviewees saw this as such a small reduction in bikes, that missing and stolen bikes are not considered a problem.

\subsection{Further questions}
After the interview we tried to contact Aalborg Kommune again, but despite several attempts, it was not possible to reach anyone.
This made further questions impossible.
