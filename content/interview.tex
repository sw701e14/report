\section{Interview with Aalborg Kommune}
The \citybike project is maintained by Aalborg Kommune.
In order to get information about \citybike, an interview was conducted with three people involved with the project.

The interview was carried out October 10th, 2014, as a semi-structured interview.
The interviewees were:
\begin{itemize}
\item Brian Høj; Responsible for the \citybike project.
\item Jesper; Member of the \citybike project.
\item Anne Mette; Member of the original team which implemented the CIVITAS-ARCHIMEDES project\cite{aalborgbycyklenbagcyklen}.
\end{itemize}

\noindent There were two overall goals of the interview:
\begin{enumerate}
\item To collect facts about how the current system.
\item To determine if the project members themselves already had improvements in mind.
\end{enumerate}

\subsection{Results} \label{interview:goals}
The following section is based on the Danish summary of the interview, found in \cref{interviewReferat}.

\paragraph{Overall goal}
The intentioned use of the bikes are single, short trips.
This means that the bikes are not meant to be used for longer periods of time, or to completely replace a personal bike.

\paragraph{Users and usage}
There is no target group for the usage of the bikes.
The bikes are currently not tracked so they were not able to say anything about the actual usage.
The only information they have is what they themselves observe when they walk around town and what citizens report.

\paragraph{Future plans}
The interviewees have before thought about installing GPS chips in the bikes to receive data about their traffic.
They would really like to have an overview of the bike usages and usage patterns.
Specifically, they would like some information about:
\begin{itemize}
\item What routes are travelled?
\item How long a distance has been travelled in this trip?
\item When was the bike used?
\item Who uses the bikes?
\end{itemize}

\paragraph{Rent/Booking}
It has never been the intention of the project to be able to rent or book a bike, because it would hamper the spontaneous use of the bikes.
If all the problems with booking could be solved in a reasonable manner, they would not rule out the future of such a system.

\paragraph{Missing bikes}
According to Traffic \& Roads, Aalborg Kommune\cite{cykelplanlaegning} there was a 11\% loss of bikes from 2009 to 2011 (237 to 210 bikes).
According to the interviewees there now, fall 2014, remain 200 of the original 237 bikes from summer 2009.
The interviewees saw this as such a small reduction in bikes, that missing and stolen bikes are not considered a problem.

\subsection{Conclusion}
The interview did not give much information about the users or usage of the system.
The target users are therefore considered to be anyone living in or currently situated in Aalborg.
Additionally, no concrete conclusions can be made about where the bikes are used the most.

The interviewees did have some restrictions and other wishes.
They did not like the idea of a renting/booking system, as they saw it to be too restricting.
They did, however, think it would be useful to track the bikes' users and usage, possibly via GPS.

\paragraph{Further contact}
After the interview we tried to contact Aalborg Kommune again, but despite several attempts, it was not possible to come into contact with them.
