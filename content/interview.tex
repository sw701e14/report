\section{Interview with Aalborg Kommune}
The Aalborg Bycyklen project is maintained by Aalborg Kommune.
In order to get information about the Aalborg Bycyklen project, an interview was conducted with three people involved with the project.

The interview was carried out November 10th, 2014, as a semi-structured interview.
The interviewees were:
\begin{itemize}
\item Brian Høj; Responsible for the Aalborg Bycykel project.
\item Jesper; Member of the Aalborg Bycykel project.
\item Anne Mette; Member of the original team which implemented the CIVITAS-ARCHIMEDES project.
\end{itemize}

There were two overall goals of the interview:
\begin{enumerate}
\item To find out specifics about how the current system works.
\item To determine if the project members themselves already had improvements in mind.
\end{enumerate}

\subsection{Results} \label{interview:goals}
The following section is based on the Danish resume of the interview, found in \cref{interviewReferat}.

\paragraph{Overall goal}
The intentioned use of the bikes are single, short trips.
This means that the bikes are not meant to be used for longer periods of time, or to completely replace a personal bike.

\paragraph{Users and usage}
There is no target group for the usage of the bikes.
The bikes are currently not tracked so they were not able to say anything about the actual usage.
The only information they have is what they themselves observe when they walk around town and what citizens report.

\paragraph{Future plans}
The interviewees have before thought about installing GPSs in the bikes to receive data about their traffic.
They would really like to have an overview of the bike usages and usage patterns.
Specifically, they would like some information about:
\begin{itemize}
\item What routes are travelled?
\item How long a distance has been travelled in this trip?
\item When was the bike used?
\item Who uses the bikes?
\end{itemize}

\paragraph{Rent/Booking}
It has never been the intention of the project to be able to rent or book a bike, because it would hamper the spontaneous use of the bikes.
If all the problems with booking could be solved in a reasonable manner, they would not rule out the future of such a system.

\paragraph{Missing bikes}
According to Traffic \& Roads, Aalborg Kommune\cite{cykelplanlaegning} there was a 11\% loss of bikes from 2009 to 2011 (237 to 210 bikes).
According to the interviewees there now, fall 2014, remain 200 of the original 237 bikes from summer 2009.
The interviewees saw this as such a small reduction in bikes, that missing and stolen bikes are not considered a problem.

\subsection{Conclusion}
It is still unclear whether a target user group exists for the project, so it is more safe to say that the target group is anyone living in or visiting Aalborg.
The system is meant to be as simple as possible, meaning that no renting or booking system should be developed.

Aalborg Kommune would also like some statistics of the users and usage of the Aalborg Bycyklen.

\paragraph{Further contact}
After the interview we tried to contact Aalborg Kommune again, but despite several attempts, it was not possible to reach anyone.
