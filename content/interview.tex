\section{Interview 10-10-2014}
The following is paraphrased from \cref{interviewReferat}.

\paragraph{Structure and participants}
The interview was a semi-structured interview. 
It was conducted by group SW707E14 and group SW705E14.
The interviewees were:
\begin{itemize}
\item Brian Høj; Responsible for the Aalborg Bycykel project.
\item Jesper; Employee at Aalborg Bycykel.
\item Anne Mette; Member of the original team which implemented the CIVITAS-ARCHIMEDES project.
\end{itemize}


\paragraph{Goals and needs}
There is no target group for the usage of the bikes.
The only goals are that the bikes should be used for single trips, not for longer periods time, or in place of a personal bike.
\alexander{Wirte to Brian: ``Why was city bikes implemented as a step in the ACHIMEDES project?''}


\paragraph{Users and usage}
The bikes are currently not tracked so they were not able to say anything about the actual usage.
The only information they have is what they themselves observe when they walk around town and what citizens report.

One of the interviewees have observed many bikes at schools and universities.
Therefore they suppose the average user is young.
They further interpret it as the bikes are used as a replacement for a personal bike.


\paragraph{Future plans}
The interviewees have before thought about installing GPSs in the bikes to receive data about their traffic.
They would really like to have an overview of the bike usages and usage patterns.
Additional they would like some information about:
\begin{itemize}
\item What routes are traveled?
\item How long a distance has been traveled in this trip?
\item When was the bike used?
\item Who uses the bikes?
\end{itemize}
\alexander{Write to Brian ``What data to they want exactly?''}


\paragraph{Rent/Booking}
It has never been the intention of the project to be able to rent/book a bike, due to it would hamper the spontaneous use of the bikes.
They were interested in the idea of rent/booking, but they were very wary of not hampering the spontaneous use of the bikes.
If all the problems with booking could be solved in a reasonable manner, they would not rule out the future of such a system.


\paragraph{Missing bikes}
According to Traffic \& Roads, Aalborg Kommune\cite{cykelplanlaegning} there was a 11\% loss of bikes from 2009 to 2011 (237 to 210 bikes).
According to the interviewees there now, fall 2014, remain 200 of the original 237 bikes from summer 2009. 
Given this small reduction in bikes, missing and stolen bikes are not considered a problem.