\bruno{More depth?}
\paragraph{Faster Data Structures}
To optimize the API, a more complex data structure should be introduced.
E.g. the process of checking whether bike locations are within a hotspot would be improved.
R-trees\cite[Section 25.3.5.3]{database_system_concepts} is a data structure that could be chosen for that purpose.

\paragraph{API Access}
The API developed in this project could be used for different purposes and users.
Therefore it would be rather practically to introduce different user access levels.
E.g. there could be the following access levels:
\begin{itemize}
\item Administrator (facilitator of the API - access to everything)
\item User (standard access)
\item Bike maintainer (access to e.g. all bikes location)
\end{itemize}
This change would be added in the \texttt{Controller} layer in the \texttt{Web Service} component.

\paragraph{Data Warehouse \cite{data_warehousing}}
To ease the maintenance of the bikes a data warehouse could be introduced.
A data warehouse would provide statistics of the bike usage, e.g. frequently travelled paths.
These statistics could be used for improving the physical system(Aalborg Bycyklen, \cref{aalborg_bycyklen}) and the software system.

\paragraph{Model Updating}
There are some issues regarding the process of updating the model(\cref{}\bruno{ref to 'creating markov chain' section}) and how often it should be updated.
There are many possibilities regarding this, because it is a rather complex matter.
Below is a list trying to show the factors that influence the model:
\begin{itemize}
\item Time of day
\item Weekday
\item What kind of day(holidays, working day, etc.)
\item Weather
\item etc.
\end{itemize}
A simple model would be to update it every week, and don't take the different factors into account.
A more complex model is to partition the day into mornings, afternoons and nights.
And for each of these have a model for holidays and working days.
As the example shows, to get the best results one model might not be enough.

To find the best model one needs a collection of real data for testing which model gives the best results.

\paragraph{Accuracy and Convex Hull}
The algorithm that finds the convex hull\cref{}\bruno{ref til convex hull - hvis der kommer et eksplicit afsnit ellers til komponentent hvor det er nævnt} from the clusters, does not take accuracy into account.
This means that some points that should be inside/outside the convex hull are not.
This can give some unexpected results.
When checking if a bike is inside a convex hull, the accuracy is not taking into account either.
\bruno{Hvad er løsningen udover bare at gøre det?} 