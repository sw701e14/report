\bruno{Not done at all!}
\paragraph{Faster Data Structures}
To optimize the API, a more complex data structure should be introduced.
E.g. the process of checking whether bike locations are within a hotspot would be improved.
R-trees\cite[Section 25.3.5.3]{database_system_concepts} is a data structure that could be chosen for that purpose.

\paragraph{API Access}
The API developed in this project could be used for different purposes and users.
Therefore it would be rather practically to introduce different user access levels.
E.g. there could be the following access levels:
\begin{itemize}
\item Administrator(facilitator of the API)
\item User
\item Bike maintainer
\end{itemize}
This change would be added in the \texttt{Controller} layer in the \texttt{Web Service} component.

\paragraph{Data Warehouse \cite{data_warehousing}}
To ease the maintenance of the bikes a data warehouse could be introduced.
A data warehouse would provide statistics of the bike usage, e.g. frequently travelled paths.
These statistics could be used for improving the physical system and the software system.

\paragraph{Model Updating}
This section is about the process of updating the model, described in \cref{}\bruno{ref to 'creating markov chain' section}, especially how often.
There are a lot of trade-off's.. fd,mjg  
Model for all mornings e.g.
model for all weekdays e.g.
etc.

Should the data collected go back to a week, month, year?

\paragraph{Accuracy and Convex Hull}
