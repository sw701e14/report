This chapter will reflect on how the developed system could be improved.

\section{Optimized Data Structure}
To optimize the API, a more complex data structure should be introduced.
E.g. the process of checking whether bike locations are within a hotspot would be improved.
Because as of now the implementation uses a list, which means that every element is accessed in worst case.
Therefore using another data structure would mean that some parts could be skipped.
R-trees\cite[Section 25.3.5.3]{database_system_concepts} is a data structure that could be chosen for that purpose.

\section{Model Updating}\label{reflection:model_updating}
There are some issues regarding the process of updating the model(\cref{}\bruno{ref to 'creating markov chain' section}) and how often it should be updated.
There are many possibilities regarding this, because it is a rather complex matter.
Below is a list trying to show the factors that influence the model:
\begin{itemize}
\item Time of day
\item Weekday
\item What kind of day (workday, holiday, city events, etc.)
\item Weather conditions
\item Etc.
\end{itemize}
A simple model would be to update it every week, and don't take the different factors into account.
An example of a more complex model is to partition the day into mornings, afternoons and nights.
And for each of these have a model for holidays and working days.

To find the best model one needs a collection of real data for testing which model gives the best results.

\section{Usage statistics for Aalborg Kommune}
One of the requests from the interview with Aalborg Kommune (see \Cref{interview:goals}) was that they wanted certain statistics about bike usage.

Since the data is available, as bike locations are already being continuously collected.
We would add a data-warehouse to extract the newest bike location updates and form its own data, so that it could be easy to query on it, queries such as those mentioned in \Cref{interview:goals}.

These statistics could be used for improving the \citybike system, as well as the software system.
The statistics could improve/help the following processes:
\begin{itemize}
\item Where to add more bike lanes
\item Improve cycle infrastructure
\item When to update the model(s)
\item etc.
\end{itemize}

\section{API Access}
Adding different queries for Aalborg Kommune, that should not be accessible to normal users, would require the API to be re-designed considering security measures.
Therefore it would be rather practically to introduce different user access levels.
For instance there could be the following access levels:
\begin{itemize}
\item Administrator (facilitator of the API - access to everything)
\item User (standard access)
\item Bike maintainer (access to e.g. all bikes location)
\end{itemize}
This change would be added in the \texttt{Controller} layer in the \texttt{Web Service} component.
