This section contains the main problem for this project and the form of solution required.

\subsection{Analysis Summary}
The analysis was based on the initial problem: \textit{How can the current \citybike system be improved, in order to make it more usable?}.
From the analysis, it was found that the current \citybike has several problems, limiting the overall usability (as described in \cref{aalborg_bycyklen:challenges}).

The problems identified were:
\begin{enumerate}
\item Bikes left outside stations
\item Too few stations
\item Making short stops
\item No bikes at station
\item No way of knowing when a bike will arrive
\item Broken bikes
\end{enumerate}

Additionally, certain limitations and requests were made by Aalborg Kommune, collected through the interview:

\begin{enumerate}
\item No renting/booking system
\item No specific target user group
\item Statistics about usage
\item Short period usage
\end{enumerate}

\subsection{Scope}
Due to the limited time and project members, certain restrictions will be set, as to remove focus on certain problems.

\paragraph{Making Short stops} will be completely overlooked in this project, as any solution to this problem would require some added locking mechanisms, which would conflict with the limitations set by Aalborg Kommune.

\paragraph{Statistics about usage} is something Aalborg Kommune also requested.
This, however, is deemed too far from the initial problem, as well as adding too much to the workload.

\subsection{Solution}
We have chosen the solution with the same bikes as the current system, with the addition of a GPS chip on each individual bike, and completely removing the stations and bike locks.
This is due to the restrictions set by Aalborg Kommune, requesting a simple system, in order to accommodate the unspecified target user group.

\paragraph{GPS and hotspots}
The greatest changes to the current system is the addition of the GPS and the introduction og hotspots.
Hotspots will serve as informal stations; areas of the map where bikes are often parked.
By combining these two things, we want to improve on the problems with the current system.

\bruno{We should consider restating the problems so that it is more covenient for the reader.}

\subparagraph{Problem 1} is handled by GPS, as bikes and their exact location now can be found, no matter if it is at a hotspot or not.

\subparagraph{Problem 2} is automatically handled by the generation of hotspots.
Whenever an area is a popular place to be going from/to, a hotspot should emerge and reflect this.

\subparagraph{Problem 3} was deemed out of scope.

\subparagraph{Problem 4} is handled by locating the exact location of an available bike, or looking up nearby hotspots.

\subparagraph{Problem 5} will be handled by modelling bike behaviour and predicting when a bike will be available at certain hotspot(s).

\subparagraph{Problem 6} will not be directly handled, but it is possible to infer from looking at bike locations and seeing when a bike has been immobile for a very long time, which could indicate that it either is too far away for anyone to pick it up, or broken.

\paragraph{Software}
A software solution is then to be developed, consisting of 3 main software components, and a shared database.
The structure of the solution can be seen in \cref{fig:solution_structure}.

\begin{figure}[h]
\includegraphics[width=\textwidth]{our_solution.pdf}
\caption{The overall structure of our solution}
\label{fig:solution_structure}
\end{figure}

The \textit{Location Service} continuously updates the locations of the bikes.
The \textit{Model Updater} uses the location updates to generate a model used for predictions.
The \textit{Webservice} makes available an API, in order to make platform implementations to make available the data as specified in the requirements.
A web service was chosen as to make the system as available as possible, due to the unspecific target user group.

\subsection{New problems}
In order to achieve this new system, with the addition of GPS and hotspots, new problems emerge.
The following chapters will focus on the problems:
\begin{itemize}
\item How to define and find hotspots?
\item How to model the usage and use this for predictions?
\item How to make this data available through a web service?
\end{itemize}