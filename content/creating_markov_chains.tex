\section{Generating Markov Chain}\label{sec:generatemarkov}
The Markov chain $\mathcal{M}$, described in \cref{markov}, is generated by the steps detailed later.
The algorithm takes a set of hotspots $H$ and a set of GPS data. For each hotspot in $H$ there exists a departure state $D_h$.
The result of the algorithm is a matrix with the probabilities for all possible transactions.

\begin{figure}
\begin{enumerate}
\item A $n \times n$ matrix with n being $|H| \times 2$ is created.
\item The GPS data is grouped into an array of arrays so each bike contains its respective GPS data.
\item The initial state is determined for each bike.
\item Then for each state we determine the new location for each bike.
\item[4.5] If the old state was a hotspot and the new state is not, then the state is the departure state associated with the previous hotspot.
\item The correct transition is then incremented in the matrix.
\item Each entry of each row is then divided by the sum of the row, thus normalizing it.
\end{enumerate}
\label{enum:markov}
\end{figure}