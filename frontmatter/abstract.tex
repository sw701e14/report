The purpose of this project was to improve on the already existing \citybike system, by implementing a software solution.

In order to improve on the already existing \citybike system, it was first analyzed, identifying and defining problems to be improved upon.
Based on these problems, along with information gained from an interview with Aalborg Kommune, a solution was formed.
This solution built on-top of the current \citybike system, with the addition of GPS receivers and the removal of bike stations.

Instead of bike stations, hotspots were introduced.
These hotspots were found by using a clustering algorithm DBSCAN.
Hotspot representation was then simplified by obtaining its convex hull.

Time-homogenous Markov chains, represented as a matrix, was then introduced in order to form a model of the average bike usage.
Based on this model, predictions were made about when to expect bikes at the different hotspots.

In order to make the data available to the users, a web service was implemented.

A pseudo-realistic simulation was created in order to ensure that the implementation was successful.
The results of the tests suggest that the implementation was successful.