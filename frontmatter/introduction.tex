In 2009, Aalborg Kommune implemented the CIVITAS-ARCHIMEDES project.\cite{civitas-archimedes}
As part of this project, the \citybike\footnote{In Danish: Aalborg Bycyklen} project was launched, making communal bikes available throughout the city.\cite{aalborgbycyklenbagcyklen}
In order to maximize the availability of the bikes, a simple system was chosen; practically free and very easy to use.
The only payment needed is a 20DKK deposit, which will be returned at the end of the ride.
Despite the bike usage being free, the bikes are well equipped and of good quality.\cite{cykelplanlaegning}

To further increase the likelihood of people using the bike as primary transportation, Aalborg strives to become a 'bicycle city'.
A bicycle city is a city which makes an effort to better its cycling infrastructure.\cite{cykelhandlingsplan}

Aalborg Kommune has over the years  made several efforts into achieving the aforementioned goal, amongst others they have:
\begin{itemize}
\item created a cycle highway from the city to the university\cite{cykelhandlingsplan}
\item built pedestrian safe zones between bus stops and cycle paths\cite{pedestriansafezone}
\item organized campaigns to increase awareness about cyclists\cite{cykelbycampaigns}
\end{itemize}

This increased focus on bikes as a primary source of transportation, furthers the importance of a publicly available bike system.
However, the system as it is now, has several problems in regards to this goal.
This will be the focus of this project, with a basis in the following initial problem:

\begin{center}
\textit{How can the current \citybike system be improved, in order to make it more usable?}
\end{center}