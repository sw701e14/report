In 2009, Aalborg Kommune implemented the CIVITAS-ARCHIMEDES project\cite{civitas-archimedes}.
As part of this project, the \citybike\footnote{In Danish: Aalborg Bycyklen} project was launched, making communal bikes available throughout the city\cite{aalborgbycyklenbagcyklen}.
In order to maximize the availability of the bikes, a simple system was chosen; practically free and very easy to use.
To use a bike one needs only to give a 20DKK deposit, which will be returned at the end of the ride.
Despite the bike usage being free, the bikes are well equipped and of good quality.\cite{cykelplanlaegning}

To further increase the likelihood of people using the bike as primary transportation, Aalborg strives to become a 'bicycle city'.
A bicycle city is a city which makes an effort to better its cycling infrastructure.\cite{cykelhandlingsplan}

Aalborg Kommune has over the years  made several efforts into achieving the aforementioned goal, amongst others they have:
\begin{itemize}
\item created a cycle highway from the city to the university\cite{cykelhandlingsplan}
\item built pedestrian safe zones between bus stops and cycle paths\cite{pedestriansafezone}
\item organized campaigns to increase awareness about cyclists\cite{cykelbycampaigns}
\end{itemize}

This increased focus on bikes as a primary source of transportation, is further emphasized by a publicly available bike system.
However, the system as it is now, exhibits some problems that prevents it from completely fulfilling its goal.
This will be the focus of this project, with a basis in the following initial problem:

\begin{center}
\textit{How can the current \citybike system be improved, in order to make it more usable?}
\end{center}