To counter the increasing energy and environmental issues, Aalborg Kommune implemented the CIVITAS-ARCHIMEDES project\cite{aalborgbycyklenbagcyklen}. As part of the project, city bikes were made publicly available for everyday use.

The aim of this project is to improve the city bikes, so that more people will use them hence improving the environment.

We will present a software oriented solution in form of a web service. This web service will give Aalborg Kommune the potential to track and get and overview of the city bikes and their usage patterns. The web service will further provide the possibility for users to find available bikes (in a defined radius with the user as center?).

This report will present a problem analysis, the design choices regarding our solution, how we implemented our design, how we tested our solution, and further development as well as reflection on this project.




\alexander{How about it now?
\giovanni{\#18, introduction is messy - do not solve this before later in the project}}