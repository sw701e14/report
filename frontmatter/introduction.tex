To counter the increasing energy and environmental issues, Aalborg Kommune implemented the CIVITAS-ARCHIMEDES project\cite{aalborgbycyklenbagcyklen}. As part of this project, city bikes were made publicly available for everyday use.
In order to maximize the number of users, the usage was cheap and easy but the bikes were well equipped and was of good quality\cite{cykelplanlaegning}. The only payment needed is a 20DKK deposit, which will be returned at the end of the ride.

To further increase the likelihood of people using the bike as primary transportation Aalborg strives to become a `cycle city'.
A cycle city is a city which makes an effort to better its cycling infrastructure.\cite{cykelhandlingsplan}

Aalborg Kommune has over the years 
created a cycle highway from the city to the university, \cite{cykelhandlingsplan}
built pedestrian safe zones between bus stops and cycle paths\cite{pedestriansafezone},
organized campaigns to increase awareness about cyclists\cite{cykelbycampaigns},
and much more.
\giovanni{\#18, introduction is messy - do not solve this before later in the project}