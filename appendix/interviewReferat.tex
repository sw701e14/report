\chapter{Interview 10-10-2014 - Summary}\label{interviewReferat}
Interviewede personer:
\begin{enumerate}
\item Brian Høj - Ansvarlig for Aalborg Bycyklen.
\item Jesper - Ansat hos Aalborg Bycyklen.
\item Anne Mette - Med til at starte ACHIMEDES projektet op.
\end{enumerate}

\paragraph{- Hvilket overblik har I over cyklerne? lokation, forbrug, antal, osv. Hvordan finder I ud af om der skal flere cykler i brug, og hvor cyklerne skal placeres?}
Der er i øjeblikket ingen kontrol med cyklerne og dermed heller ingen  data omkring brugen
af cyklerne. Den eneste information er gehør, altså hvad de selv observerer ude i byen,
og de opringninger de modtager fra borgerne der melder en cykel forsvundet.

\paragraph{- Får I nogle klager? Hvis ja, hvad lyder klagerne på?}
De klager de oftest får er at der ikke er nok cykler tilgængelig. De observerer også selv
at der er helt tomme stationer.

\paragraph{- Har I nogle statistikker/data om systemet vi kan få adgang til?}
Der er ingen data overhovedet. Kun anelser om hvordan cyklerne bliver brugt. En af de interviewede
observerede at der holdt mange bycykler ved skoler og universiteter, hvilket de fortolker som om
at der er mange der bruger cyklerne som privatcykler.
Hvad de fandt interessant er at få mere data om brugen af cyklerne. Dette kunne fås v.h.a. GPS, ruter, strækning, stationer, tid, 
stilstand, samme bruger?

\paragraph{- Kan vi bruge det samme layout som http://www.aalborgbycyklen.dk m.h.t. Copyright?}
De mener at det er dem der ejer siden, og vil lige undersøge om ikke også de har rettighederne.

\paragraph{- Hvad er jeres egentlige behov? Hvilke mål har i med projektet og hvilke restriktioner er blevet påsat jer for at opnå disse mål?}
De er usikre på om det er turister eller borgere der er målgruppen for brugen af cyklerne, men de ligger vægt på at den ikke
bliver brugt som privatcykler. For eksempel hvis en person ønsker at bruge den resten af året ud.
Den større politiske opmærksomhed på bycykler sammen med tendensen til at der bliver afsat flere penge af til sådanne projekter
leder dem til at konkludere at det er et system der skal udvides på, og gøres mere moderne.

\paragraph{- Har i nogle udvidelser/videreudviklings planer til projektet?}
De er interesseret i at tracke cyklerne med GPS og indsamle data om cyklernes brug, men de har ikke nogle konkrete planer om hvad
der skal laves og hvordan man sikrer at GPS'en ikke bliver stjålet fra cyklen.

\paragraph{- Har i haft nogle tanker om reservering af bycyklerne?}
Der var ingen tanker i den retning med systemet, men idéen om at booke en cykel virkede interessant for dem. De var åbne overfor
muligheden, men kunne også ulemperne ved et sådant system, da det kan blive for restriktivt i forhold til brugeren.
For eksempel, hvad sker der hvis brugeren slet ikke dukker op til sin booking? Hvor længe i forvejen skal man kunne booke en cykel?
Hvornår må systemet konstatere at brugeren ikke dukker op?
De ser brugen af cyklerne som en spontan handling fra brugeren og måske ikke så meget planlagt.
 
\paragraph{- Har i lyst til at være vores kunde? Højst 3 møder mere i løbet af semester.}
Ja, de var villige til at lave en forsøgsperiode på en uge eller mere, hvor eksempelvis ti cykler blev udstyret med GPS, så man kunne
studere cyklernes færden.
Aftalen lyder, at vi tre grupper snakker sammen om vores videre arbejde og kontakt med dem som kunder.

\subsection{Rå notater:}

\begin{itemize}
\item Oprindeligt 200 cykler men kun 140 cykler ude nu. Resten er i depot.
\item Et privat firma der står for driften. En dyr ordning. Der mangles sponsor indtægter.
\item Ingen GPS på cyklerne. Simpelt system.
\item Har overvejet at se på noget mere avanceret. Måske nogle nye cykler? tænker de
Århus, har andre cykler hvor de ikke har problemer med sponsorer (store reklamer på hjulene). Færre driftsomkostninger
\item Der forsvinder ikke mange cykler. (ikke noget særligt)
\item Cyklerne samles ind i oktober. De finder allesammen, men de kan godt finde mærkelige steder. De kommer ud igen først i april.
\item De har problemer med at cyklerne ikke bliver fundet af sig selv men rapporteret forsvundet af brugerne.
\item De kunne godt tænke sig at have kontrol af cyklerne (med GPS har de overvejet)
\item De vil også gerne vide mønstrene af brug af cyklerne.
\item Der findes ingen data for brug af cyklerne. Det eneste der sker ved cyklerne er vedligeholdelse.
\item De vil gerne have turister til at bruge cyklerne
\item Målgruppen er: turist eller borger (ikke så meget hvem der er, men at den ikke bliver brugt som sin private cykel)
\item GPS ideen lyder spændende for dem.
\item De er bevidste om at det er en dårlig ordning.
\item De er interesseret i data og statistikker om hvordan det bliver brugt.
\item Booking er også interessant, men GPS delen er særlig interessant.
\item Hvor længe holder de stille? Hvor meget er de i brug?
\item Deres egne observationer er at der holder mange ved skoler og universiteter og kan måske konkludere ud fra det at de bliver brugt til privat brug.
\item De tror mere på de spontane brug af bycyklerne, og mener at det bliver meget restriktivt med booking.
\end{itemize}