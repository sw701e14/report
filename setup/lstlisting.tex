% Her er en liste over navnene på de forskellige styles
% C#: csharp
% F#: fsharp

% 
% Listings kan refereres vha. \cref{}
\crefname{listing}{code example}{code example}
\Crefname{listing}{Code example}{code examples}
% 

%Algoritmer i cref
\crefname{algocf}{algorithm}{algorithm}
\Crefname{algocf}{Algorithm}{Algorithms}
%Algoritmelinjer i cref
\crefalias{AlgoLine}{line}%

\makeatletter
\let\cref@old@stepcounter\stepcounter
\def\stepcounter#1{%
  \cref@old@stepcounter{#1}%
  \cref@constructprefix{#1}{\cref@result}%
  \@ifundefined{cref@#1@alias}%
    {\def\@tempa{#1}}%
    {\def\@tempa{\csname cref@#1@alias\endcsname}}%
  \protected@edef\cref@currentlabel{%
    [\@tempa][\arabic{#1}][\cref@result]%
    \csname p@#1\endcsname\csname the#1\endcsname}}
\makeatother
%

% Angivelse af navn på listings
\renewcommand\lstlistingname{Code example}
\renewcommand\lstlistlistingname{Code example}

\lstdefinestyle{standard}
{
	frame=shadowbox,
	framesep=5pt,
	rulecolor=\color{blue!40!black},
	rulesepcolor=\color{white!93!black},
	numbers=left,
	basicstyle=\ttfamily,
	numberstyle=\tiny,
	numberfirstline=true,
	%numberblanklines=false,
	stepnumber=1,
	numbersep=9pt,	
	captionpos=b,
	escapeinside={(*}{*)},
	breaklines=true,
	tabsize=4,
	language=c
}

\lstset{style=standard}

\lstdefinestyle{c}
{
	style=standard
}

\lstdefinestyle{csmall}
{
	style=c
}

\lstdefinestyle{csharp}
{
	style=standard,
	language=[Sharp]C
}
\lstdefinestyle{csharpsmall}
{
	style=csharp
}
\lstdefinestyle{fsharp}
{
	language=[Sharp]F,
	frame=lr,
	rulecolor=\color{blue!80!black}
}
\lstdefinestyle{fsharpsmall}
{
	style=fsharp,
	basicstyle=\ttfamily\footnotesize
}

